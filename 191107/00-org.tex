% !TeX root = 191107.tex
\section{Организационное}
\begin{frame}[t]{Зачем курс}
	\begin{itemize}
		\item Писать много-много кода, ошибаться
		\item Перенимать опыт через code review:
			\begin{enumerate}
			\item Пишете код
			\item \label{improve} Говорим, как улучшить
			\item Вы улучшаете
			\item См. п.\ref{improve}
			\end{enumerate}
		\item Развить <<чувство прекрасного>> с точки зрения кода
		\item Рассказать про разные стили/парадигмы кода
		\item Показать, какие бывают полезные технологии
	\end{itemize}
\end{frame}

\begin{frame}[t]{Зачёт и проверка}
	\begin{itemize}
		\item Правила зачёта подробно есть на CSCWiki
		\item Посещаемость неважна, экзамена нет, только домашки
		\item Надо написать код. Оцениваются:
			\begin{enumerate}
			\item Объективно: корректность и \textit{точное} соответствие заданию "--- 40\%
			\item Субъективно: идиоматичность, красота кода (<<стиль>>) "--- 40\%
			\item Субъективно: покрытие тестами "--- 20\%
			\end{enumerate}
		\item Все домашки блокирующие
		\item Обычно: одна домашка "--- одна тема (темы почти независимы)
		\item В некоторых домашках будет несколько подзаданий
		\item На домашку около недели
		\item Детали зависят от группы
			\begin{itemize}
				\item Насколько интерактивная проверка
				\item Что после срока сдачи
				\item Как именно можно досдавать до конца семестра
			\end{itemize}
	\end{itemize}
\end{frame}

\begin{frame}[t]{Рекомендации}
	\begin{itemize}
		\item \textbf{Курс на <<работу руками>> и чтобы вы сами что-то делали}
		\item Обсуждать можно, но не стоит сдавать чужой код или набирать код за кого-нибудь
		\item
			Любая критика и жалобы на жизнь также приветствуются.
			Особенно если есть предложения <<как лучше>>.
		\item
			Делитесь тайными знаниями не только с товарищами, но и со мной.
			Тогда я знаю, что я упустил на паре.
	\end{itemize}
\end{frame}

\begin{frame}[t]{Структура занятий}
	\begin{itemize}
		\item На лекциях: теория, демонстрация, в любой момент вопросы
		\item В четверг появляется условие домашнего задания, можно читать и готовить вопросы
		\item На практиках: повтор теории своими руками, подготовка к домашке, живое общение
	\end{itemize}
\end{frame}
