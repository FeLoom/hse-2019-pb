% !TeX root = 191107.tex
\section{Парадигмы программирования}
\begin{frame}[t,fragile]{Императивная программа}
\begin{minted}{cpp}
int main(int argc, char* argv[]) {
    int i = 0, val = 0, flag = 0;
  next_arg: i++;
    if (i >= argc) goto parsed;
    if (!strcmp(argv[i], "--flag")) goto process_flag;
    if (!strcmp(argv[i], "--val")) goto process_val;
    goto next_arg;
  process_flag: flag = 1; goto next_arg;
  process_val: i++; assert(i < argc);
    val = atoi(argv[i]);
    goto next_arg;
  parsed:
    printf("val=%d, flag=%d\n", val, flag);
    return 0;
}
\end{minted}
\end{frame}

\begin{frame}[t,fragile]{Структурная программа}
\begin{minted}{cpp}
int main(int argc, char* argv[]) {
    int val = 0, flag = 0;
    for (int i = 1; i < argc; i++) {
      if (!strcmp(argv[i], "--flag")) {
        flag = 1;
      } else if (!strcmp(argv[i], "--val")) {
        i++;
        assert(i < argc);
        val = atoi(argv[i]);
      }
    }
    printf("val=%d, flag=%d\n", val, flag);
    return 0;
}
\end{minted}
\end{frame}

\begin{frame}[t,fragile]{Псевдоопределение}
	<<Парадигма программирования>> "--- это соглашение о том, \ldots
	\begin{itemize}
	\item \ldots как выражать алгоритм в коде
	\item \ldots как записывать алгоритм
	\item \ldots какие конструкции языка (не) использовать (\verb!goto!)
	\item \ldots как структурировать программу
	\item \ldots что является \textit{примитивом}
	\item \ldots как примитивы \textit{композировать}
	\item \ldots какие есть способы \textit{абстракции}
	\end{itemize}
\end{frame}

\begin{frame}[t,fragile]{Зачем изучать}
	\begin{itemize}
	\item Язык программирования "--- это инструмент, есть плюсы и минусы
	\item Парадигма "--- это инструмент, есть плюсы и минусы
	\item	На одном языка можно писать в разных парадигмах, иногда можно смешивать.
	\item Выбирайте инструмент под задачу
	\end{itemize}
	Бегло познакомимся с кучей инструментов для кругозора.

	На других курсах будете углубляться.
\end{frame}

\begin{frame}[t,fragile]{Упражнение}
	Чему равно следующее выражение?
	\[
	23 \cdot 2 + 59 \cdot 2 + 18 \cdot 2 +
	23 \cdot 5 + 59 \cdot 5 + 18 \cdot 5 +
	23 \cdot 3 + 59 \cdot 3 + 18 \cdot 3
	\]
	\pause
	\[
	23 \cdot (2 + 5 + 3) + 59 \cdot (2 + 5 + 3) + 18 \cdot (2 + 5 + 3)
	\]
	\pause
	\[
	(23 + 59 + 18) \cdot (2 + 5 + 3)
	\]
	\pause
	\[
	100 \cdot 10 = 1000
	\]

	Мораль: про простые конструкции думать проще и надёжнее.
\end{frame}

\begin{frame}[t,fragile]{Упражнение}
	Где баг в программе?
\begin{minted}{cpp}
if (!strcmp(ops, "+++")) a = x + y + z + t;
if (!strcmp(ops, "++-")) a = x + y + z - t;
if (!strcmp(ops, "+-+")) a = x + y - z + t;
if (!strcmp(ops, "-++")) a = x - y + z + t;
if (!strcmp(ops, "--+")) a = x - y - z + t;
if (!strcmp(ops, "-+-")) a = x - y + z + t;
if (!strcmp(ops, "+--")) a = x + y - z - t;
if (!strcmp(ops, "---")) a = x - y - z - t;
\end{minted}
	\only<2->{
	\begin{itemize}
	\item Какая вероятность напороться на баг в случайном тесте?
	\item Сколько случаев для $n$ операций?
		\only<3->{$\mathcal O(2^n)$}
	\item Сколько символов написать верно для $n$ операций?
		\only<3->{$\mathcal O(n \cdot 2^n)$}
	\end{itemize}
	}
\end{frame}

\begin{frame}[t,fragile]{Как решать механически}
\begin{minted}{cpp}
if (!strcmp(ops, "+++")) a = x + y + z + t;
if (!strcmp(ops, "++-")) a = x + y + z - t;
if (!strcmp(ops, "+-+")) a = x + y - z + t;
if (!strcmp(ops, "-++")) a = x - y + z + t;
if (!strcmp(ops, "--+")) a = x - y - z + t;
if (!strcmp(ops, "-+-")) a = x - y + z + t;
if (!strcmp(ops, "+--")) a = x + y - z - t;
if (!strcmp(ops, "---")) a = x - y - z - t;
\end{minted}
\end{frame}

\begin{frame}[t,fragile]{Как решать механически}
\begin{minted}{cpp}
if (!strcmp(ops, "+++")) a = x + y + z + t;
if (!strcmp(ops, "++-")) a = x + y + z - t;
if (!strcmp(ops, "+-+")) a = x + y - z + t;
if (!strcmp(ops, "+--")) a = x + y - z - t;
if (!strcmp(ops, "-++")) a = x - y + z + t;
if (!strcmp(ops, "-+-")) a = x - y + z + t;
if (!strcmp(ops, "--+")) a = x - y - z + t;
if (!strcmp(ops, "---")) a = x - y - z - t;
\end{minted}
\end{frame}

\begin{frame}[t,fragile]{Как решать механически}
\begin{minted}{cpp}
// ops[0] == '+'

if (!strcmp(ops, "+++")) a = x + y + z + t;
if (!strcmp(ops, "++-")) a = x + y + z - t;
if (!strcmp(ops, "+-+")) a = x + y - z + t;
if (!strcmp(ops, "+--")) a = x + y - z - t;
// ops[0] == '-'

if (!strcmp(ops, "-++")) a = x - y + z + t;
if (!strcmp(ops, "-+-")) a = x - y + z + t;
if (!strcmp(ops, "--+")) a = x - y - z + t;
if (!strcmp(ops, "---")) a = x - y - z - t;
\end{minted}
\end{frame}

\begin{frame}[t,fragile]{Как решать механически}
\begin{minted}{cpp}
if (ops[0] == '+') {

  if (!strcmp(ops, "+++")) a = x + y + z + t;
  if (!strcmp(ops, "++-")) a = x + y + z - t;
  if (!strcmp(ops, "+-+")) a = x + y - z + t;
  if (!strcmp(ops, "+--")) a = x + y - z - t;
} else { 

  if (!strcmp(ops, "-++")) a = x - y + z + t;
  if (!strcmp(ops, "-+-")) a = x - y + z + t;
  if (!strcmp(ops, "--+")) a = x - y - z + t;
  if (!strcmp(ops, "---")) a = x - y - z - t;
}
\end{minted}
\end{frame}

\begin{frame}[t,fragile]{Как решать механически}
\begin{minted}{cpp}
if (ops[0] == '+') {
  int xy = x + y;
  if (!strcmp(ops + 1, "++")) a = xy + z + t;
  if (!strcmp(ops + 1, "+-")) a = xy + z - t;
  if (!strcmp(ops + 1, "-+")) a = xy - z + t;
  if (!strcmp(ops + 1, "--")) a = xy - z - t;
} else { 
  int xy = x - y;
  if (!strcmp(ops + 1, "++")) a = xy + z + t;
  if (!strcmp(ops + 1, "+-")) a = xy + z + t; // ??!
  if (!strcmp(ops + 1, "-+")) a = xy - z + t;
  if (!strcmp(ops + 1, "--")) a = xy - z - t;
}
\end{minted}
\end{frame}

\begin{frame}[t,fragile]{Или сразу догадаться}
	По определению на самом деле мы вычисляем так:
	\[
		x \pm y \pm z \pm t = ((x \pm y) \pm z) \pm t
	\]
\begin{minted}{cpp}
int xy;
if (ops[0] == '+') xy = x + y
else               xy = x - y;
if (ops[1] == '+') xyz = xy + z
else               xyz = xy - z;
if (ops[2] == '+') a = xyz + t
else               a = xyz - t;
\end{minted}
	Если смогли выделить общие куски:
	\begin{itemize}
	\item Надо разобрать $O(n)$ случаев
	\item Надо написать только $O(n)$ символов
	\end{itemize}
	Пример из домашки: \texttt{loadtext savetext}
\end{frame}
