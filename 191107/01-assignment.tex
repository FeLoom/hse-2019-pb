% !TeX root = 191107.tex
\section{Домашнее задание}

\begin{frame}[t]{Основное}
	\begin{itemize}
		\item Описание и правила сдачи "--- на CSCWiki
			\begin{itemize}
				\item \textbf{Укажите правильную тему письма}
				\item Попыток бесконечно, но скорость проверки конечна
			\end{itemize}
		\item Надо написать код и тесты
		\item Куча блоков по автопроверкам:
			\begin{itemize}
			\item Расстановка пробелов, имена переменных...
			\item На практике научат запускать
			\item Не прошли хоть один "--- получили 30\% за домашку
			\end{itemize}
	\end{itemize}
\end{frame}

\begin{frame}[t]{Первая домашка: grep}
	\begin{itemize}
		\item Две итерации:
			\begin{itemize}
			\item Надо написать свою утилиту grep
			\item Через неделю \textit{внезапно} потребуется дописать
				ещё функциональность
			\end{itemize}
		\item Если код стильный, добавить будет просто
		\item Если код нестильный, добавить будет сложно
		\item Можно и нужно сдавать заранее
		\item Срок первой итерации "--- следующая пятница, 22:59
	\end{itemize}
\end{frame}

\begin{frame}[t,fragile]{Тонкость: регулярные выражения}
	Стандартный способ описать множество строк, отличаются в математике и программировании.

	\begin{center}
	\begin{tabular}{|l|c|}
	\hline
	Альтернативы
		& \{\t{a}, \t{b}, \dots, \t{z}\} \\
		& \t{a|b|c|}\dots\t{|z} \\
		& \t{[a-z]} \\\hline
	Последовательность
		& \{\t{a0}, \t{a1}, \t{b0}, \t{b1}\} \\
		& \t{[a-b][0-1]} \\\hline
	Скобочки
		&\t{(a|b)(0|1)} \\\hline
	Повторы 
		& \t{-?[0-9]+} \\
		& \t{[a-z][a-zA-Z0-9\_]*} \\\hline
	Сокращения\footnote{Совсем специфичны для конкретных реализаций}:
		& \t{[0-9]}, \t{[a-zA-Z0-9\_]} \\
		& \t{\textbackslash{}d}, \t{\textbackslash{}w} \\\hline
	\end{tabular}
	\end{center}

	Поиграть: \href{https://regex101.com}{regex101.com}.
\end{frame}
